\chapter{Introduction}
\section{Motivation}
A neural machine translation (NMT\nomenclature[nmt]{NMT}{Neural machine translation}) model usually has trouble translating sentences that differ in genre, register, or theme from the sentences used for training the model. This is a common limitation of data-driven machine learning methods, whose performance is guaranteed by assuming that the training and testing distributions are identical. Therefore, to achieve high performance in a given domain, we must carefully tailor the NMT model to that domain. The problem of tailoring an NMT model to a target domain is referred to as the \SB{domain adaptation problem}. Two factors make this problem complex, including the scarcity of training data from the target domain and the catastrophic forgetting problem of the deep models. The lack of training data drives us to leverage parallel data from other domains to train our NMT models. The neural network-based models need many data to optimize their parameters. Therefore, we usually have to adapt our NMT model to the target domain using lots of out-of-domain data and a small amount of data from the target domain. Second, several approaches to adapting an NMT model by finetuning it with the in-domain data only make its performance very brittle to the out-of-domain test. This problem is referred to as \SB{catastrophic forgetting} in the neural network literature. The neural models tend to perform dramatically worse in previous tasks after being trained to perform their current tasks. In real applications, we usually aim to significantly improve the performance in the target domain and the robustness of NMT models with respect to previous training domains.

The domain problem is addressed to some extent by domain adaptation approaches as long as the testing domain is known before training. However, in many applications, such as online translation on the web, the text to translate can be from any domain. We consider this situation "non domain-deterministic testing" in our review \ref{chap:mdmt-review}. In this situation, we could build domain-expert models for the source domains and combine their predictions during the inference \citep{Saunders19domain} to get a domain-adapted translation on the fly. Besides the mixture of model paradigm, several methods use context to improve the translation of similar sentences. However, these methods do not guarantee domain robustness against the out-of-domain text.

Moreover, an MT engine has to translate text from many domains whose genre and topic are highly variable in real applications. The strategy "one domain / one model" will cost us largely when the number of domains explodes. Therefore, developing a multi-domain machine translation (MDMT) system is essential for the MT business.

This thesis aims to provide a complete overview of the multi-domain adaptation problem in machine translation and study the approaches to adapting an NMT system to many domains with a small computation and storage cost.

\section{Contributions}
In this thesis, our contributions are as follows.

First, we provide a generalized framework of the machine translation (multi-)domains adaptation problem. We point out four main situations in the domain mismatch problem. We provide a complete match between each case and its feasible methods.

Second, we provide a new multi-criteria evaluation for MT (multi-)domain methods. We reevaluate a large set of methods with our proposed experimental settings corresponding to our proposed criteria.

Thirdly, we propose, evaluate and analyze a cheap MT multi-domain method, which uses sparse word embedding with domain-specified units. The method is much cheaper than residual adapters. Besides an improvement in some mild multi-domain settings, the method can handle a growing number of domains. We extend the idea of sparse representation to higher layers of an NMT model. We demonstrate an equivalent performance of the method to several strong MDMT methods. We propose a novel analysis method for word embeddings, which identifies domain-agnostic and domain-specific tokens by observing the variation of K nearest neighbors of one token while changing its domain.

Our fourth contribution is a thorough study on the use of the residual adapters in (multi-)domain adaptation. We demonstrate its practicality and strong performance in a multi-domain MT setting consisting of a large set of domains with unbalanced sizes. We propose different regularization methods to avoid overfitting on low-resourced domains. Finally, we propose two more robust variants that are robust with respect to the domain label errors and slightly reduce the computation cost.

Next, we study dynamical sampling strategies for multi-domain machine translation. We show that those methods improve the data sampling from the mix of in-domain corpora with respect to the heuristic fixed sampling strategy. Furthermore, we demonstrate their effectiveness in several particular settings such as uni-domain adaptation, bi-domain adaptation, and unseen domain adaptation.

Finally, we study two popular paradigms for unknown test domains which rely on text retrieval. Those techniques search for the most similar translations and incorporate this additional information into the prediction of an NMT model. We demonstrate their efficacy and their weakness as well. Besides, we propose a simple variant that slightly improves the performance of previous techniques and is able to leverage synthesis translations.

\section{Outline}
The structure of the thesis is as follows.

Chapter \ref{chap:nmt-review} is a review of neural machine translation. The chapter provides basic knowledge in the text processing for neural machine translation systems, neural architectures for machine translation, the training, and the inference procedures of neural machine translation models.

Chapter \ref{chap:mdmt-review} reviews the literature of (multi-)domain adaptation in machine translation. We introduce four main sub-problems of (multi-)domain adaptation and provide an overview of the approaches for each sub-problem.

The remainder of the thesis consists of our original work. Chapter \ref{chap:revisiting} proposes a novel multi-criteria evaluation for multi-domain machine translation systems. We reevaluated a large set of model-centric approaches using a relatively large collection of domains. Chapter \ref{chap:ldr} proposes a (multi-)domain NMT system with cheap computation cost by using a sparse word embedding that nullifies a number of domain-specific units. Chapter \ref{chap:res} proposes several approaches to regularize the residual adapters \citep{Bapna19simple} in the (multi-)domain setting. In this chapter, we propose two variants of the residual adapter that allow us to modularize the domain-agnostic and domain-specific representation and to improve the performance of the adapted NMT model against out-of-domain examples. In chapter \ref{chap:mdac}, we study multiple dynamic sampling approaches for training the NMT model in the (multi-)domain setting. We propose a novel method that automatically iteratively adapts the sampling distribution to any pre-determined testing distribution. Chapter \ref{chap:priming} discusses and reimplements different retrieval-based methods for unknown test domains. We carefully analyze their performance in many domains and demonstrate their strengths and issues in terms of latency, errors. 

Finally, in chapter \ref{chap:conclusion}, we draw conclusions in the current state of the development of MDMT. We recall a need for effort to achieve the long-term goal of this approach.















































































